\chapter{安装学习环境}

\section{Kotlin优点}
Kotlin是一门现代新型编程语言,易读、简洁、安全、能与Java互相调用,开发安卓软件的首选。

\section{装Java开发包}
下载并安装JDK版本 $\geq$ 1.8,阿里巴巴的如[龙井8或11](http://dragonwell-jdk.io/)。
在命令行中查看java及javac版本,验证安装成功。
\begin{listing}[H]
    \begin{minted}[linenos,frame=lines]{bash}
        java -version
        javac -version
    \end{minted}
    \caption{查看java及javac版本}
    \label{lst:java版}
\end{listing}

\section{装集成开发环境}
下载并安装\href{https://www.jetbrains.com/idea/download}{IntelliJ IDEA}。

\subsection{在IDEA中写第一个程序}\label{子节:集程1} 
运行IDEA,按如下步骤创建、制造并运行第一个Kotlin程序:
\begin{enumerate}
    \item 在欢迎对话框中选“Create New Project (创建新工程)”
    \item 在“New Project (新工程)”页左框选Kotlin,右框选Kotlin/JVM,点击Next去下页
    \item 名工程为“问候”,依喜好自行修改其它选项
    \item 点击Finish,IDEA即开始创建一个上述设置的Kotlin工程
    \item 在工具条中点击锤子制造工程,点击向右绿三角运行
\end{enumerate}


\subsection{在IDEA中使用REPL} 
接上节,从菜单Tools -> Kotlin -> Kotlin REPL 打开REPL——Read Evaluate Print Loop(读算印环),
往REPL键入代码\ref{lst:首用REPL}然后按CTRL+回车运行。
\begin{listing}[H]
    \begin{minted}[linenos,frame=lines]{kotlin}
        fun 欢迎词(){
            println("您好,欢迎您使用Kotlin。")
        }
        欢迎词()
    \end{minted}
    \caption{首次使用Kotlin REPL}
    \label{lst:首用REPL}
\end{listing}

REPL也可以在Kotlin的命令行交互环境内使用,运行命令kotlinc即进入该环境。
\begin{listing}[H]
    \begin{minted}[linenos,frame=lines]{kotlin}
        $ kotlinc
        Welcome to Kotlin version 1.5.0 (JRE 1.8.0_292-b10)
        Type :help for help, :quit for quit
        >>> fun 欢迎(){
        ... println("您好,欢迎使用kotlin。")
        ... }
        >>> 欢迎()
        您好,欢迎使用kotlin。
\end{minted}
\caption{Kotlin命令行REPL}
\label{lst:Kotlin命令行}
\end{listing}

\section{李白诗歌}

《静夜思》\citep{李白:静夜思}
\begin{verse}
床前明月光,疑是地上霜。
举头望明月,低头思故乡。    
\end{verse}
 

%\bibliographystyle{unsrtnat}  %% 若用数字引用需将 natbib选项设为"numbers"  
\bibliographystyle{dcu}
\bibliography{参考}

