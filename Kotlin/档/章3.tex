\chapter{函式}

打开IDEA, 在第\ref{子节:集程1}节中创建的“问候”程序将自动打开。
\section{新建并运行Kotlin文件}
在工程面板中,右键点击“问候(工程名)-> src -> main -> kotlin”目录,从弹出菜单中选 New->Kotlin Class/File,然后从弹出第对话框中选File,文件名取“欢迎”,IDEA会在上面目录内生成空文件“欢迎.kt”。在“欢迎.kt”中键入如下代码。

% \begin{listing}[H]
    \begin{minted}[linenos,breaklines,frame=lines,autogobble,python3,
        style=friendly]{kotlin}
        fun main(程参: Array<String>) {
        println("欢迎")
        }

    \end{minted}
% \caption{新建Kotlin文件}
% \label{lst:新建Kotlin文件}
% \end{listing}

所有Kotlin函式都返回点啥。没指定返回类型无return语句的Kotlin main函式会默认返回Kotlin.Unit,称为Kotlin式“无返回值”。

中IDEA中,点击main函式左边的绿三角,选“Run '欢迎.kt'”即可运行该函式,在志窗(log)中可见其输出。

\section{传参给Kotlin程序}
中IDEA中,点击main函式左边的绿三角,选“Edit Run Configuration...”,在弹出对话框的“Program Arguments”文本框内输入参数,多个参数以空格分开,如“小李 20 小刘 23”,再点击OK按钮完成。

Kotlin程序参数被IDEA以Array<String>类型传给main函式,如下:
% \begin{listing}[H]
    \begin{minted}[linenos,breaklines,frame=lines,autogobble,python3,
        style=friendly]{kotlin}
        fun main(程参: Array<String>) {
        println("欢迎, ${程参[0]}")
        }

    \end{minted}
% \caption{新建Kotlin文件}
% \label{lst:新建Kotlin文件}
% \end{listing}


然后按前节方法运行该函式,在志窗(log)中可见其输出。

\section{多物有值}
Kotlin多数语句是表达式而有值,如无显式返回的函式其实返回Kotlin.Unit。if语句返回真假,但循环语句例外,不是表达式而无值。
% \begin{listing}[H]
    \begin{minted}[linenos,breaklines,frame=lines,autogobble,python3,
        style=friendly]{kotlin}
        fun main(程参: Array<String>) {
            val 单元 = println("欢迎, ${程参[0]}") // => 欢迎, 小李
            println(单元) // => kotlin.Unit
          
            val 得分 = 98
            val 通过 = if (得分 > 70) true else false
            println(通过) // => true
            println("我${if (得分>70) "通过" else "挂科"}了。") // => 我通过了。
        }
    \end{minted}
% \caption{新建Kotlin文件}
% \label{lst:新建Kotlin文件}
% \end{listing}

\section{函式示例}
Kotlin函式由两部分构成:函头——曲括号外的部分,及函体——曲括号及其内语句。
% \begin{listing}[H]
    \begin{minted}[linenos,breaklines,frame=lines,autogobble,python3,
        style=friendly]{kotlin}
        import java.util.*

        fun 哪天():String{
          var 一周 = arrayOf<String>("周一","周二","周三","周四","周五","周六","周日")
          return (一周[Random().nextInt(6)])
        }
        fun 做啥(天 :String) :String{
            // when句值是满足条件分支的末尾表达式值.
          return when (天){
            "周一" -> "打篮球"
            "周二" -> "登山"
            "周三" -> "游泳"
            "周四" -> "打鼓"
            "周五" -> "唱歌"
            else -> "发呆"
          }
        }
        fun 周几做啥(){
          val 天 = 哪天()
          val 活动 = 做啥(天)
          println("我${天}${活动}")
        }
        
        fun main(程参: Array<String>) {
          周几做啥()
        }
    \end{minted}
% \caption{函式示例}
% \label{lst:函式示例}
% \end{listing}

\section{简函与函参默认值}
只含一个表达式的函式可以紧凑地写成简函,即把函体换成其表达式赋值给函头。带默认值的函参没被按位或名传值时则取其默认值。
% \begin{listing}[H]
    \begin{minted}[linenos,breaklines,frame=lines,autogobble,python3,
        style=friendly]{kotlin}
        fun 答对么(解: Double, 答案: Double) :Boolean{
            return if (解 == 答案) true else false
        }
        // 对应简函
        fun 答对(解: Double, 答案: Double) = if (解 == 答案) true else false

        fun 矩形面积(长: Double, 宽:Double) :Double{
            return 长 * 宽
        }
        // 对应简函
        fun 矩面(长: Double, 宽:Double) :Double = 长 * 宽

        fun 默参(甲: Int, 乙: Boolean = true, 丙: Double = 20.0){
            println("传入参数:甲=${甲},乙=${乙}, 丙=${丙}。")
        }

        fun main(程参: Array<String>) {
            val 长 = 4.0
            val 宽 = 5.0
            val 面积 = 矩形面积(长, 宽)
            println("长${长}宽${宽}的矩形面积为${面积}")
            println("长${长}宽${宽}的矩形面积为${矩面(长, 宽)}")

            val 答案 = 20.0
            val 结果 = 答对(19.0, 答案)
            println("你的结果${if (结果) "对了" else "错了"}。")
            println("你的结果${if (答对么(19.0, 答案)) "对了" else "错了"}。")

            // 按位传参
            默参(1)
            默参(2,false)
            默参(3,false, 11.3)
            // 按位、名传参
            默参(4, 丙=11.4)
            默参(5, 丙=11.5, 乙 = false)
            // 按名传参
            默参(丙=11.6, 甲=6)
            默参(丙=11.7, 甲=7, 乙 = false)
        }
    \end{minted}
% \caption{简函与函参默认值}
% \label{lst:简函与函参默认值}
% \end{listing}

\section{队列滤式}
队列滤式可以提取队列中满足条件的元素。

% \begin{listing}[H]
    \begin{minted}[linenos,breaklines,frame=lines,autogobble,python3,
        style=friendly]{kotlin}
        fun main(程参: Array<String>) {
            val 产品质量 = intArrayOf(53, 67, 13, 34, 88, 64)
            // 滤式默认为迫滤
            println("合格产品:" + 产品质量.filter { it >= 60 })

            // 迫滤:滤则创建完整的滤出队列,完全滤出
            val 迫滤 = 产品质量.filter { it < 60 }
            println("迫滤不合格产品:" + 迫滤)

            // 惰滤:要用到满足条件的元素时再滤出之
            val 惰滤 = 产品质量.asSequence().filter { it < 60 }
            println("惰滤不合格产品:" + 惰滤)
            val 全要 = 惰滤.toList() // 要时再滤
            println("惰滤全部结果:" + 全要)

            // 逐个要惰滤结果
            val 懒藕 = 惰滤.asSequence().map { it * -1 }
            println("懒藕:${懒藕}")

            println("懒藕第一个元素:${懒藕.first()}")
            println("懒藕出了几个:${懒藕}")
            println("懒藕第一个元素:${懒藕.elementAt(0)}")
            println("懒藕第二个元素:${懒藕.elementAt(1)}")
            println("懒藕第三个元素:${懒藕.elementAt(2)}")

            // 全要
            val 惰藕 = 惰滤.asSequence().map { it - 60 }
            println("惰藕:${惰藕.toList()}")
        }
    \end{minted}
% \caption{简函与函参默认值}
% \label{lst:简函与函参默认值}
% \end{listing}

\section{量函与参函}
函式可当变量——量函,可当参函——参函。
量函(Lambda)也称匿名函式,可以当作变量使用。因可做变量,所以可以当参数传递,称参函。带参函的函式叫高阶函式,如前节滤式(filter)和映射(map),传入其中的表达式就被隐式地转换为量函。

% \begin{listing}[H]
    \begin{minted}[linenos,breaklines,frame=lines,autogobble,python3,
        style=friendly]{kotlin}

        fun 矩形面积(长: Double, 宽:Double) :Double{
            return 长 * 宽
        }
        
        fun main(程参: Array<String>) {
            // 匿名函式(lambda),类型由kotlin隐式得出
            val 面积 = {长: Double, 宽:Double -> 长 * 宽}
            val 长 = 4.0
            val 宽 = 5.0
            println("长${长}宽${宽}的矩形面积为${面积(长, 宽)}")

            // 显式指定匿函类型
            val 矩积: (Double, Double) -> Double = {长, 宽 -> 长 * 宽}
            println("长${长}宽${宽}的矩形面积为${矩积(长, 宽)}")

            // 高阶函式(带函参的函式)
            val 求面积: (Double, Double, (Double,Double)->Double)->Unit = {
                长,宽,面积函式 -> val 面积 = 面积函式(长,宽)
                println("长${长}宽${宽}的矩形面积为${面积}")
            }
            求面积(5.0, 8.0, 矩积)
            求面积(5.0, 8.0, ::矩形面积) // 有名函式当函参

            // 若高阶函式仅最后的参数为函参,也可这样传,但会造成阅读困难
            求面积(5.0, 8.0){ 长,宽 -> 长*宽 }
        }
    \end{minted}
% \caption{简函与函参默认值}
% \label{lst:简函与函参默认值}
% \end{listing}

% 满江红·怒发冲冠\citep{岳飞:满江红}
% \begin{verse}
%     [宋] 岳飞
%     怒发冲冠,凭栏处潇潇雨歇。
%     抬望眼,仰天长啸,壮怀激烈。
%     三十功名尘与土,
%     八千里路云和月。
%     莫等闲白了少年头,空悲切。

%     靖康耻,犹未雪;
%     臣子恨,何时灭!
%     驾长车踏破贺兰山缺。
%     壮志饥餐胡虏肉,
%     笑谈渴饮匈奴血。
%     待从头收拾旧山河,朝天阙。    
% \end{verse}



%\bibliographystyle{unsrtnat}  %% 若用数字引用需将 natbib选项设为"numbers" 
% \bibliographystyle{dcu}
% \bibliography{参考}