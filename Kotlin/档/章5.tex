\chapter{集}
藕、叁、恒量、扩

\section{藕、叁}
藕——二元组,叁——三元组。

% \begin{listing}[H]
    \begin{minted}[linenos,breaklines,frame=lines,autogobble,python3,
        style=friendly]{kotlin}
        fun main(args: Array<String>) {
            // 藕
            var 国都 = "中国" to "北京"
            println("${国都.first}的首都是${国都.second}")
        
            // 叁
            val 高铁 = Triple("上海", "北京", 6.0)
            println("${高铁.first}到${高铁.second}的G106高铁只要${高铁.third}小时。")
            println(高铁.toList())
            println(高铁.toString())
        
            val 悬铁 = Triple("F106" to 6.0, "上海" to "北京", 106)
            println("悬浮列车${悬铁.first.first}共${悬铁.third}节车厢,从${悬铁.second.first}到${悬铁.second.second}只要${悬铁.first.second}小时。")
            println(悬铁.toList())
            println(悬铁.toString())
        
            // 取部件
            val (发,到,_) = 高铁
            println("发于${发},到达${到}")
        }        

    \end{minted}
% \caption{藕叁}
% \label{lst:藕叁}
% \end{listing}

\section{列、键值对}

% \begin{listing}[H]
    \begin{minted}[linenos,breaklines,frame=lines,autogobble,python3,
        style=friendly]{kotlin}
        fun main(args: Array<String>) {
            // 列
            val 果园产量 = listOf<Double>(200.0, 123.3, 66.98)
            println("果园总产量:${果园产量.sum()}公斤")
        
            val 水果 = listOf<String>("苹果","葡萄","橙")
            println("水果名字长度和:${水果.sumOf { s: String -> s.length }}")
        
            println(水果)
            for (果 in 水果.listIterator()){
                println(果)
            }
            // 键值对
            val 众考生 = hashMapOf<String, Int>("张三" to 123456, "李四" to 22345, "王五" to 32345)
            println("众考生:${众考生}")
            println("张三的考号是${众考生["张三"]}")
            println("小二的考号是${众考生["小二"]}")
            println("小二的考号是${众考生.getOrDefault("小二", 55555)}")
            println("小二的考号是${众考生.getOrElse("小二", {55555})}")
        
            var 果产集 = mutableMapOf<String, Double>("苹果" to 100.0, "葡萄" to 212.2, "香蕉" to 232.5)
            println(果产集)
            果产集.put("李", 333.3)
            println(果产集)
            果产集.replace("苹果", 500.0)
            println(果产集)
            果产集.remove("葡萄")
            println(果产集)
        }
    \end{minted}
% \caption{列键值对}
% \label{lst:列键值对}
% \end{listing}

\section{恒量}
恒量以const val定义,在编译时赋值,且只能做全局量。而val变量在运行时赋值,因此可用由函式赋值。

因恒量只能定义于全局域内,故不能定义于常类,但可定义于独类。


% \begin{listing}[H]
    \begin{minted}[linenos,breaklines,frame=lines,autogobble,python3,
        style=friendly]{kotlin}
        // 恒量只能定义为全局变量,在编译时赋值,故不能赋值以函式
        const val 圆周率 = 3.1415926
        // 在运行时赋值,可赋值以函式
        val 地球质量 = 5.9736e24
        fun 太阳系行星质量(行星:String):Double{
            return when(行星) {
                "地球" -> 5.9736e24
                "天王星" -> 8.6832e25
                else -> 0.0
            }
        }
        val 天王星质量 = 太阳系行星质量("天王星")
        object 行星质量{
            const val 地球质量 = 5.9736e24
            const val 天王星质量 = 8.6832e25
        }
        class 圆{
            // 相当Java的static
            companion object{
                const val 圆周率 = 3.1415926
                fun 周长(直径:Double):Double{
                    return 圆周率*直径
                }
                fun 面积(直径:Double):Double{
                    return 圆周率*直径*直径/4.0
                }
            }
        }
        fun main(args: Array<String>) {
            println("圆周率=${圆周率}")
            println("地球质量=${地球质量}")
            println("天王星质量=${天王星质量}")
            println("圆周率=${圆.圆周率}")
            val 直径 = 6.0
            println("直径为${直径}的周长为${圆.周长(直径)}")
            println("直径为${直径}的面积为${圆.面积(直径)}")
            println("地球质量=${行星质量.地球质量}")
        }        
    \end{minted}
% \caption{恒量}
% \label{lst:恒量}
% \end{listing}

\section{扩}
扩无需已有类源码即可扩其功能——增加成员,但扩法内只能访问其公开成员。Kotlin标准库主要用扩实现。

% \begin{listing}[H]
    \begin{minted}[linenos,breaklines,frame=lines,autogobble,python3,
        style=friendly]{kotlin}
        // 扩函
        fun String.含数():Boolean{
            return this.contains("""\d""".toRegex())
        }
        fun String.含汉字():Boolean = find { it >= 0x4e00.toChar() && it <= 0x9fff.toChar() } != null
        // 扩性
        val String.有数: Boolean
            get() = 含数()
        
        fun main(args: Array<String>) {
            println("邮编110108".含数())
            println("邮编110108".含汉字())
        }            
    \end{minted}
% \caption{藕叁}
% \label{lst:藕叁}
% \end{listing}

\section{泛型}
