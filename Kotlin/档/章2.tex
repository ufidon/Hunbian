\chapter{Kotlin基础}

\section{类型与算符}
Kotlin类似Java,属静态强类型语言,支持Java的数型如Int、Long、Float、Double,及算符+、-、*、/,可以在Kotlin的命令行交互环境中体验。注意其中的数型自动判定、整数除法、及整数实数(带小数点)混合运算结果类型。

% \begin{listing}[H]
    \begin{minted}[linenos,breaklines,frame=lines]{kotlin}
        >>> 1+3
        res0: kotlin.Int = 4
        >>> 4-1
        res1: kotlin.Int = 3
        >>> 10/2
        res2: kotlin.Int = 5
        >>> 2.0/3.0
        res3: kotlin.Double = 0.6666666666666666
        >>> 3.0*4.0
        res4: kotlin.Double = 12.0
        >>> 1+3.0
        res9: kotlin.Double = 4.0
        >>> 2.0*3
        res10: kotlin.Double = 6.0
        >>> 3/4
        res11: kotlin.Int = 0
        >>> 3/4.0
        res12: kotlin.Double = 0.75
        >>> 
    \end{minted}
% \caption{Kotlin数型与算符}
% \label{lst:Kotlin数型与算符}
% \end{listing}

Kotlin有两种量型:只读与可变。只读量用val(value,值的缩写)申明,只可赋值一次,再次赋值将出错。可变量用var(variable,可变的缩写),可多次赋值。

% \begin{listing}[H]
    \begin{minted}[linenos,breaklines,frame=lines]{kotlin}
        >>> // 申明量,类型由kotlin隐式推导
        >>> var 身高 = 172
        >>> 身高 = 180
        >>> val 小赵身高=150
        >>> 小赵身高 = 172
        error: val cannot be reassigned
        小赵身高 = 172
        ^
        >>> // 显式指定量型
        >>> var 班级人数: Int = 50
        >>> var 班级平均分: Double = 95.8        
    \end{minted}
% \caption{Kotlin量型与赋值}
% \label{lst:Kotlin量型与赋值}
% \end{listing}

Kotlin不隐转数型,需在代码中明确指定。

% \begin{listing}[H]
    \begin{minted}[linenos,breaklines,frame=lines]{kotlin}
        >>> val 身高: Int = 172
        >>> val b身高 = 身高.toByte()
        >>> val c身高 = 身高.toChar()
        >>> b身高
        res17: kotlin.Byte = -84
        >>> c身高
        res18: kotlin.Char = ¬
        >>> val 节: Byte=10
        >>> val 整1: Int = 节
        error: type mismatch: inferred type is Byte but Int was expected
        val 整1: Int = 节
                      ^
        
        >>> val 串1: String = 整1
        error: unresolved reference: 整1
        val 串1: String = 整1
                         ^
        
        >>> val 实数: Double = 节
        error: type mismatch: inferred type is Byte but Double was expected
        val 实数: Double = 节
                         ^
        
        >>> val 整2: Int = 节.toInt()
        >>> 整2
        res24: kotlin.Int = 10
        >>> val 串2: String = 节.toString()
        >>> 串2
        res26: kotlin.String = 10
        >>> val 实数: Double = 节.toDouble()
        >>> 实数
        res28: kotlin.Double = 10.0
        >>> // 大数字分组
        >>> val 一亿 = 1_0000_0000
        >>> val 深圳座机号 = 755_1234_5678
        >>> val 十六进制数 = 0x55_AA_55_aa
        >>> var 二进制数 = 0b0101_0101_1010_1010
        >>> 一亿
        res34: kotlin.Int = 100000000
        >>> 深圳座机号
        res35: kotlin.Long = 75512345678
        >>> 十六进制数
        res36: kotlin.Int = 1437226410
        >>> 二进制数
        res37: kotlin.Int = 21930
        >>> val 长整数 = 2L
        >>> 长整数
        res39: kotlin.Long = 2                
    \end{minted}
% \caption{Kotlin数型与转换}
% \label{lst:Kotlin数型与转换}
% \end{listing}

Kotlin字串由单引号‘ ’或双引号“ ”包围,用 + 拼接。字串模板内用\$\{量名\}表示其值,易读。

% \begin{listing}[H]
    \begin{minted}[linenos,breaklines,frame=lines]{kotlin}
        >>> val 鱼数 = 3
        >>> val 熊掌数 = 5
        >>> var 宣称1 = "我有${鱼数}鱼" + "${熊掌数}熊掌"
        >>> var 宣称2 = "我共有${鱼数+熊掌数}鱼与熊掌"
        >>> 宣称1
        res51: kotlin.String = 我有3鱼5熊掌
        >>> 宣称2
        res52: kotlin.String = 我共有8鱼与熊掌
        >>> // 字串模板需用双引号,单引号出错
        >>> var 宣称3 = '我共有${鱼数+熊掌数}鱼与熊掌'
        error: too many characters in a character literal ''我共有${鱼数+熊掌数}鱼与熊掌''
        var 宣称3 = '我共有${鱼数+熊掌数}鱼与熊掌'
          ^
            
    \end{minted}
% \caption{字串拼接与模板}
% \label{lst:字串拼接与模板}
% \end{listing}

\section{贰值与比较式}
贰值(Boolean)为真假或对错相对立的两个值,常用于比较式的结果。比较算符因此也称贰符,如==、!=、<、>、<=和>=。多个比较式用\&\&(与)、||(或)、!(非)连接,组成复杂条件,用作if/else(若/否)句的条件。

% \begin{listing}[H]
    \begin{minted}[linenos,breaklines,frame=lines]{kotlin}
        >>> // 若句
        >>> val 心跳频率 = 70
        >>> if (心跳频率 in 45..105){ println("心跳频率正常") }
        心跳频率正常
        >>> // 若否句
        >>> val 白球数 = 11
        >>> val 黑球数 = 13
        >>> if (白球数 > 黑球数){
        ...     println("白球比黑球多。")
        ... } else {
        ...     println("白球比黑球少。")
        ... }
        白球比黑球少。
        >>> // 串联若否句,多分支条件式
        >>> val 得分 = 97
        >>> if (得分 >= 90){
        ...     println("优")
        ... } else if (得分 >= 80){
        ...     println("良")
        ... } else if (得分 >=70){
        ...     println("中")
        ... } else {
        ...     println("差")
        ... }
        优
        >>> // when句,类似Java的switch句,可用范围作条件
        when(得分){
        ...    in 90..100 -> println("优")
        ...    in 80..89 -> println("良")
        ...    in 70..79 -> println("中")
        ...    else -> println("差")
        ...}
        优
    \end{minted}
% \caption{若否句}
% \label{lst:若否句}
% \end{listing}

\section{可空变量}
可空变量可取空值(null),Kotlin变量默认不可空。
% \begin{listing}[H]
    \begin{minted}[linenos,breaklines,frame=lines]{kotlin}
        >>> // Kotlin变量默认不可空。
        >>> var 班级人数: Int = null
        error: null can not be a value of a non-null type Int
        var 班级人数: Int = null
                        ^
        >>> // 用类型?定义可空变量
        >>> var 得分: Double? = null
        >>> 得分
        res96: kotlin.Double? = null
        >>> 得分 = 98.9
        >>> 得分
        res98: kotlin.Double? = 98.9
        >>> // 用若否句判断变量值为空
        >>> var 支付宝金额 = 12
        >>> if (支付宝金额 != null){
        ...     支付宝金额 -= 6
        ... }
        >>> // 用?判断可空变量值空否?不空则调用成员方法dec()
        >>> 支付宝金额 = 支付宝金额?.dec()
        >>> 支付宝金额
        res105: kotlin.Int = 5
        >>> // 不空则调用成员方法dec(),空则取?:(elvis算符)后面的值,这里是0
        >>> 支付宝金额 = 支付宝金额?.dec() ?: 0
        >>> 支付宝金额
        res107: kotlin.Int = 4
        >>> // 指针或引用的非空判符!!可把任何量转换为非空型,若其值为空则抛出异常NullPointerExceptions
        >>> var 名字:String? = null
        >>> val 名长 = 名字!!.length
        java.lang.NullPointerException
        >>> 名字 = "张三"
        >>> val 名长 = 名字!!.length
        >>> 名长
        res112: kotlin.Int = 2

    \end{minted}
% \caption{可空变量}
% \label{lst:可空变量}
% \end{listing}

\section{队、列及循环}
队(array)和列(list)是Kotlin两组合型,类似Java的队和列。
% \begin{listing}[H]
    \begin{minted}[linenos,breaklines,frame=lines,autogobble,python3,
        style=friendly]{kotlin}
        // 1. 造列
        // listOf<T>(...)造的列不可更改
        var 宠物 = listOf<String>("鹦鹉","猫","蜥蜴","壁虎")
        println(宠物) // => [鹦鹉, 猫, 蜥蜴, 壁虎] 
        
        // mutableListOf<T>(...)造的列可以更改
       var 水果 = mutableListOf<String>("苹果","李","梨","樱桃")
        println(水果) // => [苹果, 李, 梨, 樱桃]
        水果.add("柚")
        println(水果) // => [苹果, 李, 梨, 樱桃, 柚]
        水果.removeAt(0)
        水果.remove("梨")
        println(水果) // => [李, 樱桃, 柚]
        水果.remove("梨") // => res1: kotlin.Boolean = false 

        // 2. 造队
        // 混型队
        val 家禽 = arrayOf("鸡","鸭","鹅", 3)
        println(家禽) // => [Ljava.lang.Object;@5cb88b6b        
        println(java.util.Arrays.toString(家禽)) // => [鸡, 鸭, 鹅, 3] 

        家禽[0] // => res4: kotlin.Any = 鸡        
        家禽[1]="鸟"
        error: type mismatch: inferred type is String but Nothing was expected
        家禽[1]="鸟"
            ^

        家禽.get(2) // => res8: kotlin.Any = 鹅        
        家禽.size // => res9: kotlin.Int = 4

        // 纯型队
        val 整数队1 = intArrayOf(5,3,4) // 列举造队
        val 整数队2 = intArrayOf(6,8,12)
        var 合并 = 整数队1 + 整数队2
        println(合并[4]) // => 8

        var 偶数 = Array<Int>(6){it*2} // 公式法造队
        println(偶数)
        println(java.util.Arrays.toString(偶数))
        [Ljava.lang.Integer;@15d18b61[0, 2, 4, 6, 8, 10]

        // 嵌套
        val 年龄 = intArrayOf(18, 20, 21)
        var 名字 = listOf<String>("张三", "李四", "王五")
        val 学生 = listOf(年龄, 名字, 3, "三班")
        println(学生) => [[I@51bba775, [张三, 李四, 王五], 3, 三班]

        // 遍访队列元素
        var 驴友群 = arrayOf("张三", "李四", "王五")
        for (驴友 in 驴友群){
            print(驴友+" ")
        } // => 张三 李四 王五         

        for ((号,名) in 驴友群.withIndex()){
            print("第${号}号驴友叫${名} ")
        } // => 第0号驴友叫张三 第1号驴友叫李四 第2号驴友叫王五 

        // 域与间隔
        for (数 in 3..7){
            print("${数} ")
        } // => 3 4 5 6 7 

        for (数 in 9 downTo 1 step 3){
            print("${数} ")
        } // => 9 6 3
        
        for (字母 in 'x'..'z'){
            print("${字母} ")
        } // => x y z 
        
        for (字 in '不'..'丑'){
            print("${字} ")
        } // => 不 与 丏 丐 丑       

        for (字 in '不'..'丑' step 2){
            print("${字} ")
        } // => 不 丏 丑 
        
        // 3. 循环
        var 数 = 9
        while (数 < 10){
            数++
            print("她笑了${数}次 ")
        } // => 她笑了10次
        
        do {
            数--
            print("她笑了${数}次 ")
        } while (数 > 10) // => 她笑了9次
        
        repeat(2){
            print("她又笑了${数}次 ")
        } // => 她又笑了9次 她又笑了9次
          
        // 注. var及val声明的队和可变列都可以更改其内容,但只有var的可改成引用别的队列,而val的不能
    \end{minted}
% \caption{队列}
% \label{lst:队列}
% \end{listing}





满江红·怒发冲冠\citep{岳飞:满江红}
\begin{verse}
    [宋] 岳飞
    怒发冲冠,凭栏处潇潇雨歇。
    抬望眼,仰天长啸,壮怀激烈。
    三十功名尘与土,
    八千里路云和月。
    莫等闲白了少年头,空悲切。

    靖康耻,犹未雪;
    臣子恨,何时灭!
    驾长车踏破贺兰山缺。
    壮志饥餐胡虏肉,
    笑谈渴饮匈奴血。
    待从头收拾旧山河,朝天阙。    
\end{verse}


%\bibliographystyle{unsrtnat}  %% 若用数字引用需将 natbib选项设为"numbers" 
\bibliographystyle{dcu}
\bibliography{参考}
