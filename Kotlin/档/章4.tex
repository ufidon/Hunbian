\chapter{对象编程}

\section{概念}
类、对象、接口、继承、封装、属性、方法、包

类成员为其属性与方法的统称,其可见域用见词限定:
\begin{itemize}
    \item private: 本类内可见
    \item protected:本类及其子类可见
    \item public:无见词成员默域,到处可见
    \item internal:编译单元的文件内可见,编译单元是多个文件编译成的库或程序。
\end{itemize}

\section{造类}
重用代码组织在包内,先造包。

在IDEA工程面板中,右键点击“问候(工程名)->src->main-kotlin”,从弹出菜单中选“New -> Package”,往弹出对话框的文本框内键入“礼品”做包裹名。

再右键点击包裹名“礼品”,从弹出菜单中选“New -> Kotlin Class/File”,在弹出对话框选Class,文本框内键入“瓷蛙”做类名。IDEA生成名为“瓷蛙.kt”的文件,输入如下代码

% \begin{listing}[H]
    \begin{minted}[linenos,breaklines,frame=lines,autogobble,python3,
        style=friendly]{kotlin}
        package 礼品
        import java.awt.Color

        class 瓷蛙 (var 颜色:Color = Color.RED, var 重量:Double = 50.0, var 价格:Double = 32.9){

            // 可有一个或多个初始化模块
            init {
                println("初始化1: 主构函式调用一次")
            }
            init {
                println("初始化2:主构函式调用一次")
                println("主构函式的参数:\n"+"\t颜色:${颜色.toString()}\n"+
                    "\t重量:${重量}克\n"+"\t价格:\$${价格}\n")
            }
            // 一个或多个次构函,每个次构函必须先调用主构函一次,直接地——this(),或间接地——调用别的次构函
            // 不建议使用次构函,应使用主构函集中初始化
            constructor(高度:Double):this(){
                this.高度 = 高度
            }

            // var定义可变属性,val定义只读属性
            var 高度:Double = 1.0
            var 保费:Double // Kotlin隐式为属性生成设法set()与取法get(),也可显式定义
                get() = 价格*保率
                set(保费) {保率 = 保费/价格}
            var 保率:Double = 1.0


            // 方法
            override fun toString():String{
                return ("瓷蛙描绘:\n"+"\t颜色:${颜色.toString()}\n"+
                    "\t重量:${重量}克\n"+"\t价格:\$${价格}\n"+"\t高度:${高度}\n")
            }

            fun 瓷蛙描绘(){
                println(toString())
            }
            fun 快递保费(){
                println("保费:${保费}\t 保率:${保率}")
            }
        }

    \end{minted}
% \caption{新建Kotlin文件}
% \label{lst:新建Kotlin文件}
% \end{listing}

创建主函式来使用“瓷蛙类”。右键点击包裹名“礼品”,从弹出菜单中选“New -> Kotlin Class/File”,在弹出对话框选File,文本框内键入“main.kt”做档名,输入如下代码

% \begin{listing}[H]
    \begin{minted}[linenos,breaklines,frame=lines,autogobble,python3,
        style=friendly]{kotlin}
        package 礼品
        import java.awt.Color
        
        fun 买瓷蛙(){
          // 用默认构函创建对象
          val 红瓷蛙 = 瓷蛙()
        
          // 使用对象的方法
          红瓷蛙.瓷蛙描绘()
        
          // 修改对象的属性
          红瓷蛙.重量 = 500.0
          红瓷蛙.价格 = 123.99
        
          println(红瓷蛙)
        
          // 用主构函法1造对象
          val 黑瓷蛙 = 瓷蛙(颜色 = Color.BLACK)
          黑瓷蛙.瓷蛙描绘()
        
          // 用主构函法2造对象
          val 蓝瓷蛙 = 瓷蛙(颜色 = Color.BLUE, 价格 = 100.99)
          蓝瓷蛙.瓷蛙描绘()
        
          // 用次构函造对象
          val 黄瓷蛙 = 瓷蛙(高度 = 2.3)
          黄瓷蛙.瓷蛙描绘()
        }
        
        fun main(程参: Array<String>){
          买瓷蛙()
        
          val 金瓷蛙 =瓷蛙(价格 = 2999.99)
          金瓷蛙.快递保费()
          金瓷蛙.保费 = 8999.0
          金瓷蛙.快递保费()
        }

    \end{minted}
% \caption{新建Kotlin文件}
% \label{lst:新建Kotlin文件}
% \end{listing}

类可以包含一个或多个初始化(init)模块,它们被主构函自动调用,因此它们可以使用主构函的参数。初始化模块使用的变量必须先定义。

\section{继承与子类}
Kotlin的类默认不能被继承,其成员默认不能被重载,都需标以open才行。

\section{抽象类与接口}
抽象类及接口用来定义公共属性和行为:
\begin{itemize}
    \item 抽象类与接口都不能例化。
    \item 抽象类有构函,默认open,无需显式标注。但其属性与方法默认为非抽象,无需子类实现而直接使用。显式标以abstract的方法和属性则需子类实现。
    \item 抽象类有构函而接口无。
    \item 子类只能继承一个抽象类,但可继承多个接口。
\end{itemize}

\section{接口委托}
纯属性接口委托给单对象来实现,纯行为接口委托给类。

\section{数据类}
Kotlin数据类类似C结构,但它们是对象,赋值时只赋引用。

% \begin{listing}[H]
    \begin{minted}[linenos,breaklines,frame=lines,autogobble,python3,
        style=friendly]{kotlin}
        data class 点(var 横:Double = 0.0, var 纵:Double = 0.0){
            fun 显(){
                println("座标 =(${横},${纵})")
            }
        
            override fun toString(): String {
                return "(${横},${纵})"
            }
        }
        fun main(args: Array<String>) {
            val 原点 = 点()
            原点.显()
            println("原点座标 = ${原点}")
        
            val 横点 = 点(5.0)
            val 叠原点 = 原点
            println("横点 == 原点:${if (横点 == 原点) "是" else "不"}")
            println("叠原点 == 原点:${if (叠原点 == 原点) "是" else "不"}")
            println("叠原点原点指同物:${if (叠原点 === 原点) "是" else "不"}")
        
            // 复制
            val 子点 = 横点.copy()
            println("横点 == 子点:${if (横点 == 子点) "是" else "不"}")
            println("横点子点指同物:${if (横点 === 子点) "是" else "不"}")
        
            // 解构
            val (横, _) = 横点
            println("横标 = ${横}")
        }        
    \end{minted}
% \caption{数据类}
% \label{lst:数据类}
% \end{listing}

\section{独物、枚举、闭类}
独物只有一个,冠以object。

枚举有名、号、值。

闭类只能在其所处文件内继承。

% \begin{listing}[H]
    \begin{minted}[linenos,breaklines,frame=lines,autogobble,python3,
        style=friendly]{kotlin}
        // 1. 独类
        interface 帝{
            var 号:String
        }
        object  始皇:帝{
            override var 号: String= "秦始皇"
        }
        // 2. 枚举
        enum class 色(val 值: Int) {
            红(0xFF0000), 绿(0x00FF00), 蓝(0x0000FF);
        }
        enum class 向(val 角度: Double){
            东(0.0), 南(-90.0), 西(180.0), 北(90.0);
        }
        // 3. 闭类
        sealed class 哺乳动物类{
            override fun toString(): String {
                return "我是哺乳动物。"
            }
        }
        class 灵长动物类:哺乳动物类(){
            override fun toString(): String {
                return "我是灵长动物"
            }
        }
        class 有蹄类:哺乳动物类(){
            override fun toString(): String {
                return "我是有蹄动物"
            }
        }
        
        fun main(args: Array<String>) {
            // 1. 独物
            println("${始皇}的值:${始皇.号}")
            // 2. 枚举
            println("红:名——${色.红.name},号——${色.红.ordinal},色——${色.红.值}")
            println("北:名——${向.北.name},号——${向.北.ordinal},向——${向.北.角度}")
            // 3. 闭类
            val 动物:哺乳动物类 = 有蹄类()
            when(动物){
                is 有蹄类 -> {
                    println("有蹄类")
                    println(动物)
                }
                is 灵长动物类 -> println("灵长动物类")
                else -> println("哺乳动物类")
            }
        }               
    \end{minted}
% \caption{数据类}
% \label{lst:数据类}
% \end{listing}